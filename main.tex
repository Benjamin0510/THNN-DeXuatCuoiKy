% Thêm vào preamble nếu chưa có:
\usepackage[absolute,overlay]{textpos}
\usepackage{graphicx}

% Slide mở đầu
\begin{frame}[plain,noframenumbering]
\placelogofalse

% --- Logo trường ở góc trái trên ---
\begin{textblock*}{3cm}(0.3cm,0.3cm) % (x, y) tính từ góc trái trên slide
    \includegraphics[width=2.5cm]{logo.png}
\end{textblock*}

\vspace{1cm} % cách lề trên

% --- Tên môn học ---
\centering
{\LARGE \textbf{Thực hành nghề nghiệp}}\\[0.5cm]

% --- Tên đề tài ---
{\large \textbf{Đề xuất đề tài cuối kỳ:}}\\[0.3cm]
{\Large \textbf{Phát triển ứng dụng web quản lý khách sạn}}\\
{\large \textit{(Web-based Hotel Management Application Proposal)}}\\[1.2cm]

% --- Thông tin hướng dẫn ---
{\normalsize \textbf{Giảng viên hướng dẫn:} TS Trần Sơn Hải}\\[0.2cm]
{\normalsize \textbf{Trợ giảng:} Lê Thanh Thoại}\\[0.2cm]
{\normalsize \textbf{Thực hiện bởi:} Nhóm 5}

\end{frame}


\placelogotrue

% Slide giới thiệu thành viên
\begin{frame}
    \frametitle{Giới thiệu thành viên}
    \begin{itemize}
        \item Họ và tên: Nguyễn Văn A
        \item MSSV: 12345678
        \item Lớp: K62CNTT
        \item Email: nguyenvana@example.com
    \end{itemize}
\end{frame}

% Slide tóm tắt nội dung chính
\begin{frame}
    \frametitle{Tóm tắt nội dung chính}
    \tableofcontents
\end{frame}

% 1. Lý do chọn đề tài
\section{Lý do chọn đề tài}
\begin{frame}
    \frametitle{Lý do chọn đề tài}
    Viết lý do tại sao bạn chọn đề tài ở đây.
\end{frame}

% 2. Mục tiêu và sản phẩm đầu ra
\section{Mục tiêu và sản phẩm đầu ra}
\begin{frame}
    \frametitle{Mục tiêu và sản phẩm đầu ra}
    Trình bày các mục tiêu chính và sản phẩm dự kiến.
\end{frame}

% 3. Giải pháp đề xuất
\section{Giải pháp đề xuất}
\begin{frame}
    \frametitle{Giải pháp đề xuất}
    Nêu ý tưởng giải pháp kỹ thuật hoặc tổ chức.
\end{frame}

% 4. Kiến thức và công nghệ sử dụng
\section{Kiến thức và công nghệ sử dụng}
\begin{frame}
    \frametitle{Kiến thức và công nghệ sử dụng}
    Danh sách công nghệ và kiến thức nền tảng được áp dụng.
\end{frame}

% 5. Tài nguyên sử dụng
\section{Tài nguyên sử dụng}
\begin{frame}
    \frametitle{Tài nguyên sử dụng}
    Trình bày các tài nguyên như phần mềm, phần cứng, dữ liệu.
\end{frame}

% 6. Phân công công việc và kế hoạch thực hiện
\section{Phân công và kế hoạch}
\begin{frame}
    \frametitle{Phân công công việc và kế hoạch thực hiện}
    Trình bày lịch trình, biểu đồ Gantt, ai làm gì, khi nào.
\end{frame}

% 7. Khả thi và rủi ro dự kiến
\section{Khả thi và rủi ro}
\begin{frame}
    \frametitle{Khả thi và rủi ro dự kiến}
    Đánh giá mức độ khả thi, các rủi ro, cách xử lý.
\end{frame}

% 8. Tài liệu tham khảo
\section{Tài liệu tham khảo}
\begin{frame}[allowframebreaks]
    \frametitle{Tài liệu tham khảo}
    \printbibliography
\end{frame}

% Slide kết thúc và lời cảm ơn
\begin{frame}
    \frametitle{Lời cảm ơn}
    Xin chân thành cảm ơn thầy/cô và các bạn đã lắng nghe!
\end{frame}
